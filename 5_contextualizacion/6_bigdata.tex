\subsection{Big Data}
Las bases de datos basadas en columnas están fuertemente relacionadas con el campo  \textit{Big Data}. Esto se puede entender mejor gracias a lo expuesto por \textcite{saeed2020}, los cuales destacan el hecho de que hoy en día se genera una cantidad de información enorme, debido por ejemplo a las publicaciones en redes sociales como Instagram, las millones de transacciones. Entonces, las bases de datos relacionales tienen limitaciones para poder manejar todo ese aumento, y es por eso que surgen las bases columnares, para intentar dar solución a dicho problema.

	 También, los mismos autores del párrafo anterior mencionan algunas de las características del \textit{Big Data} y cómo este tipo de bases pueden ayudar. Cuando se habla de este término, ya se refiere a que los datos son medidos en grandes cantidades, desde terabytes hasta incluso zettabytes. Luego se tiene la variedad, debido a que la información puede venir de muchísimas fuentes y por último está la velocidad a la cual se transmiten esos datos.

	Así que los aspectos anteriores buscan ser manejados por la bases de datos basadas en columnas, y gracias a \textcite{saeed2020} se pueden saber algunas de las características que ayudan a esto la capacidad de una alta compresión de los datos, usando distintos algoritmos para que sean más eficientes. De igual manera, cada columna funciona como su propio índice y los datos al mantenerse ordenados, ayudan a ahorrar espacio, así como que en consultas analíticas sólo se necesitan leer columnas específicas, lo cual evita leer innecesarias.

Como se pudo observar anteriormente, el rol de las bases de datos basadas en columnas en el manejo de \textit{Big Data} es sumamente fundamental, y gracias a todas las particularidades que posee, es hoy en día posible usar y consultar grandes volúmenes de datos.