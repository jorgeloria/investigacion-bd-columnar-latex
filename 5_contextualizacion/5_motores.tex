\subsection{Motores Disponibles}
Existen diversos motores de bases de datos orientados a columnas. Uno de ellos es \textit{Clickhouse}, el cual gracias a \textcite{schulze2024}, podemos saber algunas de sus características. Una de ellas es que es de código abierto, orientado a \textit{OLAP} y que se enfoca en atender muchas consultas de manera simultánea con una latencia baja, y esto lo hace utilizando algunas técnicas de poda o asignar de buena forma los recursos. Luego, tiene la capacidad de adaptarse a distintos formatos, sistemas para lectura y escritura, así como tener un despliegue flexible, el cual se puede adaptar a distintos tipos de hardware. Esta base de datos se divide en tres capas: la capa de integración, la de almacenamiento y la de procesamiento de \textit{queries} (en la cual usa muchas técnicas de optimización para estas).

	Asimismo, otro ejemplo de estas es \textit{Apache Cassandra}, la cual es explicada por \textcite{dourhri2021}, el cual indica que Cassandra tiene su propio lenguaje llamado CQL (\textit{Cassandra Query Language}), posee una capacidad flexible para tener distintos esquemas, se pueden tener muchos servidores de estos en un clúster y además tiene la característica de agrupamiento de datos, con el cual se pueden unir valores que tengan algún criterio en común. Otro aspecto importante es el modelado de datos de Cassandra, ya que este se compone de distintas fases que son necesarias para que pueda ser bien optimizado, empezando por el modelado de datos conceptual, luego el mapeo de datos de este al modelo lógico y finalmente terminar en el modelo físico de datos.

	Luego, se tiene a \textit{Amazon Redshift}, el cual gracias a lo detallado por \textcite{borra2024}, se pueden saber algunas de sus peculiaridades, como tener un procesamiento paralelo muy intenso con la arquitectura \textit{MPP}, orientado a operaciones con \textit{OLAP} y posee una capacidad increíble para adaptarse a cambios abruptos en las cargas de trabajo. En cuanto a su arquitectura, se caracteriza por tener el almacenamiento \textit{Redshift Managed Storage} (RMS), un nodo líder, y los otros nodos se dividen en particiones para facilitar el paralelismo, y posee una red interna muy potente para que haya una comunicación eficiente con los nodos. Además, permite integrarse fácilmente con otros servicios de AWS y posee flexibilidad en cuanto al manejo de costos.

	También, como indica \textcite{husen2021} está \textit{Google BigQuery}, la cual es una plataforma en la nube, que tiene una escala enorme de petabytes e igualmente está enfocado en \textit{OLAP}. Además, no se necesita el manejo directo de los servidores y se requiere de \textit{ANSI SQL} para manejar las consultas y como se mencionó, al tener esa gran escala, su rendimiento para queries es rápido y en cuanto a su arquitectura, separa lo que es la parte computacional y de almacenamiento.

	Finalmente, otro de estos sistemas de bases de datos es \textit{MariaDB ColumnStore}. Gracias a lo expuesto por \textcite{ilic2022}, entre sus características está que posee un procesamiento enorme para el paralelismo, orientado a una escalabilidad de gran capacidad, incluso hasta petabytes como en el caso de \textit{Google BigQuery} y unido con las características propias de hacer la organización de datos por columnas, hace que el rendimiento sea excepcional incluso cuando hay muchos datos.

	Como se mencionó anteriormente, existen diversas bases orientadas a columnas, cada una con sus particularidades, pero teniendo en común aspectos como lo es una gran capacidad de compresión, escalabilidad sobresaliente, optimizaciones para queries y varias de estas orientadas a \textit{OLAP}.
