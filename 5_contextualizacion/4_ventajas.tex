\subsection{Ventajas y desventajas}
 \subsubsection{Ventajas}
 Las bases de datos orientadas a columnas poseen una serie de ventajas, como menciona \textcite{abadi2013} una de ellas es la compresión de datos, una razón es a que existen varios valores repetidos (por ejemplo si la columna es "provincias" es muy probable que se repitan), o también a que los tipos de datos son mucho estables, como en una columna que solo hayan números, cosa que no sucede en las filas. Entonces, los algoritmos de compresión que se usan resultan más efectivos, puesto que se pueden ver como si fueran paquetes. En un caso hipotético, hay una columna que tiene posee marcas de tenis, así que al comprimirlo, van a haber patrones que se repetirán, facilitarán dicho proceso, y la compresión es una característica fundamental para mejorar el rendimiento de este tipo de bases de datos.

 Asimismo, \textcite{saeed2020} permiten aclarar otras ventajas, como lo es que las columnas al actuar por sí mismas como índices (\textit{self indexing}), entonces esto ayuda a usar menos espacio en disco. Luego, que a la hora de hacer consultas, solo trae las columnas que se le piden, así como ser más apto para OLAP (el cual \textcite{pathania2022} lo ayuda a definir como una forma de procesar datos enfocado en análisis y que funciona sobre \textit{Big Data}), esto debido a que como se realizan operaciones como sumar o promedios, entonces al trabajar con columnas, es mucho más sencillo porque solo tiene que traer una de estas y ya para hacer los cálculos, por lo que incluso para operaciones masivas los puede procesar mejor.

 Además, el autor \textcite{matei2010} también menciona otras ventajas importantes, como lo es un mejor rendimiento para tareas que sean analíticas. Esto se debe a que se necesitan menos acciones \textit{I/O}, cada columna actúa como su propio índice, y como los datos en las columnas ya se encuentran ordenados, esto permite que las agregaciones sean mucho más eficientes y por lo tanto, se puede trabajar de manera paralela, lo que aumenta el rendimiento general del sistema. También, estas bases de datos basadas en columnas requieren menos espacio en el disco, puesto que las particularidades que tienen como una alta compresión y no tener que crear índices extras (a diferencia de uno basado en filas) no solo se ahorra almacenamiento, sino un mayor procesamiento por las características mencionadas.

 \subsubsection{Desventajas}

 Por el otro lado, en cuanto a las desventajas, gracias a lo mencionado por \textcite{viswanathan2020}, una de ellas es que a la hora de insertar filas, como las columnas son guardadas de manera separada, entonces es mucho más complicado, puesto que si por ejemplo hay una fila con cinco atributos, entonces habría que buscar en cinco columnas distintas para poder agregarla, lo cual es costoso. Asimismo, en aplicaciones que involucren transacciones, ya que al componerse de varias operaciones pequeñas, entonces el costo de estar comprimiendo y separando cada vez las columnas es alto.

 De igual modo, \textcite{viswanathan2020} agrega que el tiempo de escritura es mucho mayor, puesto que en las actualizaciones, se debe ir columna por columna para modificar los registros y esto puede tornarse lento. Por último, las bases orientadas a columnas tienen una limitante en cuanto a que solo se pueden usar en sistemas que tengan que ver con \textit{analytics} o BI (\textit{Business Intelligence}), y no para operaciones que tengan que ver con transacciones o aplicaciones web.

	 Como se expuso anteriormente, las bases de datos orientadas a columnas poseen varios puntos positivos, desde compresión eficiente, lectura masiva y útiles para el \textit{Big Data}, pero también posee limitaciones en cuanto a las escrituras, ya que se vuelve mucho más lento en actualizaciones masivas de filas, así que la clave está en usarla en el área más apropiada como de operaciones de análisis.
