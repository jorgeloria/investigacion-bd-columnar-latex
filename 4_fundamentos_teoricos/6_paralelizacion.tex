\subsection{Paralelización}

 Paralelización significa dividir una tarea en subtareas más pequeñas que se ejecutan al mismo tiempo en distintos procesadores, hilos o nodos. La paralelización en bases de datos orientadas a columnas consiste en dividir el procesamiento de una consulta en múltiples subtareas que se ejecutan en paralelo; las bases orientadas a columnas al tener una organización física de los datos por columnas permite que cada una de ellas pueda procesarse de forma independiente e incluso fragmentar las columnas en segmentos más pequeños (\textit{chunks}) para distribuir la carga de trabajo según \cite{abadi2013}. Así una consulta que hace operaciones sobre diferentes atributos puede realizar cada operación en paralelo, reduciendo bastante los tiempos de respuesta para esa consulta al no tener que esperar que acabe una operación para iniciar otra.

 Para ilustrar un ejemplo podemos imaginar una consulta que calcula el promedio de la columna compras y la suma de la columna ventas, cada operación puede dividirse en fragmentos, procesarse en paralelo y al final combinarse para generar un resultado. Lo anterior resulta particularmente ventajoso en entornos OLAP, donde se deben recorrer millones de registros para obtener agregaciones tal como se observó en \cite{kanungo2017}.

 Actualmente hay sistemas como \textit{ClickHouse} que implementan paralelización de manera intensiva utilizando múltiples hilos y la división de los datos en particiones, permitiéndoles responder consultas sobre grandes volúmenes de información con latencias muy bajas según \cite{schulze2024}. Es decir que el procesamiento en paralelo es una características clave que diferencia a las bases de datos columnares de los enfoques orientados a filas, al ofrecer un mejor rendimiento en cargas de trabajo analíticas.
