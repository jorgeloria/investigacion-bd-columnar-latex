\subsection{Aspectos de consistencia y disponibilidad}
En el diseño de las bases de datos orientadas a columnas, los aspectos de consistencia y la disponibilidad toman gran importancia para garantizar un sistema robusto y de alto rendimiento, en especial en entornos distribuidos. Existe un objetivo común de garantizar un alto rendimiento en consultas, manejo de actualizaciones de datos y tolerancia a fallos. Sin embargo, los enfoques específicos del manejo de la consistencia y disponibilidad dependen del BDSM en  cuestión. Los modelos C-Store, MonetDB, VectorWise y Sybase IQ sentaron las bases de los enfoques que se utilizan hoy en día.

El sistema C-Store maneja la estrategia de replicación de datos, con el fin de garantizar disponibilidad y manejar fallos. \textcite{stonebraker2005cstore} introducen \textit{K-Safety}, una implementación que asegura que el sistema puede soportar el fallo simultáneo de \textit{K} nodos, sin perder ningún dato ni interrumpir el servicio. Esta implementación funciona guardando \textit{K+1} réplicas de datos distribuidas en diferentes nodos del clúster. Asimismo, esta replicación de datos se da a nivel de proyecciones. Una proyección es una tabla lógica almacenada físicamente como un grupo de columnas \parencite{matei2010}. Esto permite que los datos replicados tengan diferentes órdenes de clasificación, aumentando el rendimiento de las consultas y, a su vez, garantizando la disponibilidad y consistencia de los datos.

\textcite{stonebraker2005cstore} mencionan 3 posibles casos de fallo, y su recuperación usando esta implementación:
    \begin{enumerate}[1.]
        \item Si a la hora del fallo no se dio una pérdida de datos, el sistema puede volver a su versión consistente ejecutando las actualización que se quedaron en la cola de los nodos de la red.
        \item Si se da una falla catastrófica (destrucción del sistema de lectura y escritura), es necesario reconstruir ambos segmentos de otras proyecciones y \textit{join indexes} que se encuentran en el sistema.
        \item Si el sistema de escritura se daña pero el sistema de lectura se mantuvo, se recupera el sistema de escritura utilizando proyecciones que comparten la misma clave de ordenamiento y el mismo rango.
    \end{enumerate}

\textcite{abadi2013} introducen el concepto de \textit{Fractured Mirrors}, una técnica de replicación que busca mantener la disponibilidad del sistema. Funciona teniendo dos copias de datos: una en formato de filas NSM y otra en forma de columnas DSM, permitiendo que el sistema pueda atender tanto las cargas transaccionales como las consultas analíticas eficientemente.

Con respecto a la consistencia, las bases de datos orientadas a columnas tienen como desafío permitir nuevas escrituras y actualizaciones sin afectar las consultas analíticas de larga duración que ya estén en ejecución. Este problema surge debido al diseño y arquitectura de estas bases de datos. Al almacenar cada columna en un archivo físico separado, cada tupla queda distribuida en varios archivos, por lo que, por ejemplo, una sola actualización de una fila puede necesitar muchas operaciones de \textit{E/S} para lograr la modificación. \parencite{abadi2013}

Muchos de los sistemas de bases de datos orientados a columnas buscan solucionar este problema separando las operaciones de lectura de las de escritura, con el fin de mantener consistencia en los datos. En el caso de C-Store y MonetDB dividen la arquitectura en dos componentes, \textit{RS: Read Store} y \textit{WS: Write Store}. El RS contiene los datos optimizados para consultas, mientras que WS contiene las actualizaciones más recientes en memoria. Durante una consulta,  el sistema accede simultáneamente a ambos componentes para combinar los datos existentes en tiempo real, mientras que un proceso secundario integra los cambios del WS al RS con el objetivo de mantener la consistencia del sistema sin afectar las consultas analíticas \parencite{abadi2013, stonebraker2005cstore}.

Además, estos sistemas siguen un modelo de consistencia llamado \textit{Snapshot isolation}. Este modelo permite que las consultas de lectura trabajen sobre una versión estable y coherente de la base de datos, sin ser afectadas por las transacciones de escritura concurrentes \parencite{abadi2013, stonebraker2005cstore}.