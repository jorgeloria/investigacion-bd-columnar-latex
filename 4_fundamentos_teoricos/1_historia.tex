\subsection{Historia}
 La historia de las bases de datos orientadas a columnas inicia aproximadamente en 1969, con una de las primeras implementaciones de un sistema de base de datos orientado a columnas llamado \textit{TAXIR}. Según \textcite{taxir1969} TAXIR fue un sistema de recuperación de información para biología. En este sistema, los datos se agrupaban por atributos, y no por ítems y filas. Este sistema hacía posible la recuperación de datos mediante cálculos lógicos sobre los atributos, en vez de realizar una comparación secuencial de datos.

 A partir de 1970, los sistemas de bases de datos orientadas a columnas se popularizaron con la llegada de archivos transpuestos. Un archivo transpuesto es una técnica de organización de datos que almacena los datos en columnas, en vez de filas. El modelo TOD, \textit{Time Oriented Database}, fue un sistema pionero en utilizar los archivos transpuestos. TOD fue diseñado para el manejo y gestión de historiales médicos con el fin de facilitar el acceso eficiente a datos organizados en series temporales, gracias al esquema de almacenamiento basado en columnas \parencite{abadi2013}.

 En 1976, la agencia \textit{Statistics Canada} desarrolló un sistema de gestión de bases de datos con una arquitectura orientada a columnas, con el fin de gestionar eficientemente grandes volúmenes de datos estadísticos. El sistema llamado \textit{RAPID} utilizaba una estructura de archivos transpuestos, donde cada columna se almacenaba físicamente en un subarchivo separado. \textcite{turner1979} comparte la importancia de utilizar un sistema orientado a columnas, mencionando ventajas como accesos rápidos y eficientes a disco, compresión de datos bit a bit, ahorro de espacio de almacenamiento y ejecución ágil de consultas. RAPID fue un sistema de gran impacto en la historia del desarrollo de las bases de datos orientadas a columnas, ya que demostró la eficiencia de un modelo columnar para el procesamiento de datos masivos. \textcite{kanungo2017} menciona que RAPID fue compartido con otras organizaciones estadísticas durante la década de  1980, y su uso finalizó en la década de 1990.

 Otro sistema pionero de las bases de datos orientadas a columnas fue \textit{Cantor}, uno de los modelos más similares a los sistemas de almacenamiento columnar modernos. Implementado en 1984, este sistema manejaba diferentes técnicas de compresión para números enteros como la codificación delta y RLE \textit{(Run-Length Encoding)}, las cuales siguen siendo importantes en los sistemas actuales \parencite{abadi2013}.

 En 1985, se propuso el modelo DSM, \textit{Decomposition Storage Model}, predecesor del almacenamiento columnar. Este sistema proponía guardar cada columna de una tabla por separado, oponiéndose al modelo tradicional de la época NSM, \textit{N-ary Storage Model}. Cada valor de atributo se almacenaba junto con una clave sustituta para permitir la reconstrucción de la fila original. Este modelo estableció la primera diferenciación entre el almacenamiento orientado a filas y el orientado a columnas, influyendo en el desarrollo e implementación de los sistemas modernos \parencite{abadi2013}.

A partir de 1990 inició la era comercial de las bases de datos orientadas a columnas. \textit{Sybase IQ} fue uno de los primeros sistemas en comercializarse. En 1998 apareció \textit{KDB}, una base de datos columnar diseñada para el análisis de grandes volúmenes de datos de series temporales. KDB ayudó a establecer un entorno y sentar las bases para futuras implementaciones. A partir de 2005, hubo una rápida expansión con la llegada de múltiples sistemas de bases de datos orientadas a columnas, tanto comerciales como de código abierto. \parencite{whitney-history-kdb, kdb-intro-timestored}.