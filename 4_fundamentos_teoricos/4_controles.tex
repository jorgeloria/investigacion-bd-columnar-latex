\subsection{Controles de Concurrencia}
Los sistemas pioneros de las bases de datos orientadas a columnas, como C-Store, propusieron implementaciones que hoy en día son relevantes. Una de ellas es el aislamiento de instantánea (\textit{Snapshot isolation}). \textcite{stonebraker2005cstore} mencionan que mediante este método la consulta de una lectura no accede directamente al estado más reciente de la base de datos, sino a una "instantánea" que refleja el estado consistente de un punto anterior de la base de datos. A esta marca del tiempo se le refiere como \textit{High Water Mark, HWM}, la cual garantiza que todas las transacciones anteriores a ese punto ya se completaron. Este método elimina la necesidad de utilizar algún tipo de bloqueo en lecturas, mejorando la concurrencia del sistema sin comprometer la consistencia.

Para que el aislamiento de instantánea funcione, el sistema debe implementar las actualizaciones como una eliminación lógica, seguida de una inserción. Estas eliminaciones no borran físicamente los datos, sino que se marcan como inválidos en una estructura llamada \textit{Deleted Record Vector, DRV}. Esta estructura lleva un registro con una marca de tiempo en la que una tupla fue eliminada \parencite{stonebraker2005cstore}.

Las transacciones que realizan algún tipo de modificación de datos sí utilizan un control de concurrencia basado en bloqueos, como en los sistemas tradicionales. \textcite{stonebraker2005cstore} mencionan el uso del protocolo \textit{strict two-phase locking}, en donde se utilizan bloqueos sobre datos a escribir. Esto crea una tabla de bloqueos distribuida, garantizando la consistencia durante la ejecución concurrente de transacciones. Sin embargo, las operaciones de escritura son menos frecuentes en un ambiente OLAP, por lo que el impacto de bloquear estas transacciones no será tan grave.

Los mecanismos de concurrencia implementados en las bases de datos orientadas a columnas permiten mantener consistencia y un buen rendimiento en entornos analíticos, al combinar aislamiento de instantánea y bloqueos estrictos en operaciones de escritura.